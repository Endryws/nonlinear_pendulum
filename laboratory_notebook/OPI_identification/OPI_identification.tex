\labday{Identificação das órbitas periódicas instáveis imersas no atrator apresentado}

\experiment{OPIs identificadas}

Utilizando r1 = 0.3 e r2 = 0.6 foram identificadas as seguintes OPIs.
\\

\begin{center}
$\begin{array}{ccc}
	x(rad)        & y(rad/s)      & n \\  
	4.1  & -0.74  &  1\\ 
	2.1  & -2.04  &  1\\ 
	2.65 & -10.62 &  2\\ 
	6.5  & -6.0   &  2\\ 
	4.17 &  8.04  &  3\\ 
	5.95 & -0.60  &  3\\ 
	2.27 &  2.34  &  3\\ 
	2.45 & -8.06  &  4\\ 
	0.25 &  10.01 &  5\\ 
	5.46 & -13.93 &  5\\ 
	0.41 & -14.10 &  5\\ 
	0.54 &  -3.09 &  5\\ 
	5.58 &  11.31 &  5\\ 
	5.54 &  2.71  &  6\\ 
	0.94 &  6.91  &  6\\ 
	5.28 &  7.61  &  6\\ 
	0.67 &  -10.50&  6\\ 
	3.20 &  -2.38 &  6\\ 
	4.2  &  1.63  &  6\\ 
	5.48 &  -5.08 &  6\\ 
	3.16 &  1.90  &  6\\ 
	2.21 &  -4.34 &  6\\ 
	5.27 &  0.78  &  6\\ 
	1.11 &  5.52  &  6\\ 
	1.28 &  -14.23&  7\\ 
	5.09 &  12.08 &  7
\end{array}$ 
\end{center}

\experiment{Espaços de estado das órbitas identificadas}

Abaixo são apresentados os espaços de estado para cada uma das órbitas identificadas
\\



