\documentclass[idxtotoc,hyperref,openany]{labbook} % 'openany' here removes the gap page between days, erase it to restore this gap; 'oneside' can also be added to remove the shift that odd pages have to the right for easier reading

\usepackage[ 
  backref=page,
  pdfpagelabels=true,
  plainpages=false,
  colorlinks=true,
  bookmarks=true,
  pdfview=FitB]{hyperref} % Required for the hyperlinks within the PDF
  
\usepackage{booktabs} % Required for the top and bottom rules in the table
\usepackage{float} % Required for specifying the exact location of a figure or table
\usepackage{graphicx} % Required for including images
\usepackage{lipsum} % Used for inserting dummy 'Lorem ipsum' text into the template
\usepackage[portuges]{babel}
\usepackage[utf8]{inputenc}
\usepackage{pdfpages}
\usepackage{subfiles}

\newcommand{\HRule}{\rule{\linewidth}{0.5mm}} % Command to make the lines in the title page
\setlength\parindent{0pt} % Removes all indentation from paragraphs
\graphicspath{{images/}}
%----------------------------------------------------------------------------------------
%	DEFINITION OF EXPERIMENTS
%----------------------------------------------------------------------------------------

\newexperiment{example}{This is an example experiment}
\newexperiment{example2}{This is another example experiment}
\newexperiment{example3}{This is yet another example experiment}
\newexperiment{table}{This shows a sample table}
%\newexperiment{shorthand}{Description of the experiment}

%---------------------------------------------------------------------------------------

\begin{document}

%----------------------------------------------------------------------------------------
%	TITLE PAGE
%----------------------------------------------------------------------------------------

\begin{titlepage}
	\begin{center}
		\huge{Universidade Federal do Rio de Janeiro}

\vspace{10pt}
\begin{figure}[!ht]
\centering
\includegraphics[width=6cm]{mec.jpg}
\hspace{1cm}
\includegraphics[height=3cm, width=7cm]{Mecanon.jpg}
\end{figure}
        
        \vspace{85pt}
        
		\textbf{\LARGE{Caderno de Laboratório}}
		\large{\\ \textbf{Controle de um pêndulo não linear}}
		\vspace{160pt}
		
	\end{center}
	
	\begin{flushleft}
		\begin{tabbing}
			Aluno\qquad\qquad\= Endryws Medeiros Costa de Moura\\
			\>\hspace{7mm}\textbf{endmoura@poli.ufrj.br}\\
			\\
			
			Professor\> Marcelo Amorim Savi \\
			\>\hspace{-5mm}\textbf{savi@mecanica.coppe.ufrj.br}
		
	\end{tabbing}
		  
	\end{flushleft}
	
	\begin{center}
		\vspace{\fill}
		Caderno iniciado em 19 de setembro de 2017
	\end{center}
\end{titlepage}

\tableofcontents

%\mainmatter % Use Arabic numerals for page numbers

%-----------------------------------------------------------
\newpage
\section{Análise numérica do problema}

\subfile{state_space/state_space.tex}
\subfile{poincare_section/poincare_section.tex}
\subfile{OPI_identification/OPI_identification.tex}






\end{document}